\documentclass[a4paper,12pt]{article}
\usepackage{hyperref}
\usepackage{geometry}
\geometry{top=1in,bottom=1in,left=1in,right=1in}

\title{Software Requirements Specification (SRS)\\LA City Sidewalk Assessment Project}
\author{Aaren Quintero, Luis Ponce, Dillin Noriga, Jonathan Little, Nikole Cabera}
\date{April 28, 2025}

\begin{document}

\maketitle

\begin{abstract}
This document provides the Software Requirements Specification (SRS) for the LA City Sidewalk Assessment Project. It defines the functionality, performance, and constraints required for the system.
\end{abstract}

\tableofcontents
\newpage

\section{Introduction}

\subsection{Purpose}
This document outlines the Software Requirements Specification (SRS) for the LA City Sidewalk Assessment Project. It specifies the requirements for the system which aims to assess sidewalk conditions in Los Angeles and enable users to report and visualize sidewalk issues.

\subsection{Scope}
The system will allow users (residents and community workers) to submit sidewalk assessment reports and view visualized data on sidewalk conditions. The goal is to improve urban infrastructure and streamline sidewalk maintenance and repair tasks.

\subsection{Definitions, Acronyms, and Abbreviations}
\begin{itemize}
    \item \textbf{SRS}: Software Requirements Specification
    \item \textbf{API}: Application Programming Interface
    \item \textbf{GIS}: Geographic Information System
\end{itemize}

\subsection{References}
\begin{itemize}
    \item System Design Document (SDD)
    \item Software Design Specifications
\end{itemize}

\newpage
\section{Team Members}
The following individuals are responsible for the development and implementation of the LA City Sidewalk Assessment Project:

\begin{itemize}
    \item \textbf{Jonathan Little} - Project Manager, Lead Developer
    \item \textbf{Luis Ponce} - Frontend Developer
    \item \textbf{Dillin Noriga} - Backend Developer
    \item \textbf{Aaren Quintero} - Database Specialist
    \item \textbf{Nikole Cabera} - UX/UI Designer
\end{itemize}

\newpage
\section{Overall Description}

\subsection{Product Perspective}
The LA City Sidewalk Assessment system is a web-based application for reporting and visualizing sidewalk conditions. It is designed to work with existing GIS systems and provide both individual and collective insights into sidewalk conditions.

\subsection{Product Features}
\begin{itemize}
    \item \textbf{Sidewalk Assessment Form}: A form for submitting sidewalk condition reports, including an option to upload media.
    \item \textbf{Interactive Map}: A map showing sidewalk assessment data.
    \item \textbf{Search Functionality}: Users can search specific locations and view sidewalk reports.
    \item \textbf{Data Analytics}: The system will offer analytics based on user-submitted data for better decision-making.
\end{itemize}

\subsection{User Classes and Characteristics}
\begin{itemize}
    \item \textbf{Residents}: Can submit sidewalk assessments and view reports.
    \item \textbf{Community Workers}: Can view submitted data and prioritize repairs.
    \item \textbf{Admin Users}: Manage user data, edit sidewalk assessments, and handle system configuration.
\end{itemize}

\subsection{Operating Environment}
The system will be deployed as a web application and will work on major browsers such as Chrome, Firefox, and Safari. It will be hosted on a cloud platform and access will be available via HTTPS.

\newpage
\section{Functional Requirements}

\subsection{User Registration and Login}
\begin{itemize}
    \item \textbf{FR1}: The system shall allow users to register an account with basic details (name, email, password).
    \item \textbf{FR2}: The system shall allow users to log in using their registered credentials.
\end{itemize}

\subsection{Sidewalk Report Submission}
\begin{itemize}
    \item \textbf{FR3}: The system shall allow users to submit a report regarding sidewalk conditions, including a description and media upload option.
    \item \textbf{FR4}: The system shall allow users to indicate the location of the sidewalk on a map or through coordinates.
\end{itemize}

\subsection{Data Visualization}
\begin{itemize}
    \item \textbf{FR5}: The system shall display sidewalk condition data on an interactive map.
    \item \textbf{FR6}: The system shall allow users to filter sidewalk data by location, condition, and report date.
\end{itemize}

\subsection{Reporting and Analytics}
\begin{itemize}
    \item \textbf{FR7}: The system shall generate reports summarizing sidewalk conditions by neighborhood or city zone.
    \item \textbf{FR8}: The system shall provide analytics based on submitted reports to identify high-priority sidewalk issues.
\end{itemize}

\newpage
\section{Non-Functional Requirements}

\subsection{Performance Requirements}
\begin{itemize}
    \item \textbf{NFR1}: The system shall load interactive maps with sidewalk data in under 5 seconds.
    \item \textbf{NFR2}: The system shall handle up to 100 concurrent users without performance degradation.
\end{itemize}

\subsection{Security Requirements}
\begin{itemize}
    \item \textbf{NFR3}: The system shall use HTTPS to encrypt all data between the user and the server.
    \item \textbf{NFR4}: The system shall store user passwords in an encrypted format.
\end{itemize}

\subsection{Usability Requirements}
\begin{itemize}
    \item \textbf{NFR5}: The system shall be accessible on all modern browsers, including Chrome, Firefox, and Safari.
    \item \textbf{NFR6}: The user interface shall be intuitive with clear navigation and tooltips for new users.
\end{itemize}

\subsection{Reliability Requirements}
\begin{itemize}
    \item \textbf{NFR7}: The system shall be available 99\% of the time, excluding scheduled maintenance.
\end{itemize}

\subsection{Maintenance Requirements}
\begin{itemize}
    \item \textbf{NFR8}: The system shall be easily maintainable with clear and comprehensive documentation for future developers.
\end{itemize}

\newpage
\section{External Interfaces}

\subsection{User Interfaces}
\begin{itemize}
    \item \textbf{UI1}: Login screen for user authentication.
    \item \textbf{UI2}: Form for submitting sidewalk assessments, including text input and media upload functionality.
    \item \textbf{UI3}: Interactive map for visualizing sidewalk conditions, with filter options for data analysis.
\end{itemize}

\subsection{Hardware Interfaces}
The system shall be platform-independent and require no specific hardware for user access, beyond an internet-enabled device.

\subsection{Software Interfaces}
\begin{itemize}
    \item The system shall integrate with mapping and GIS APIs to provide accurate sidewalk condition visualization.
\end{itemize}

\subsection{Communication Interfaces}
\begin{itemize}
    \item The system will use HTTP/HTTPS protocols for communication between the client and server.
\end{itemize}

\newpage
\section{Legal and Ethical Considerations}

\subsection{User Data Privacy and Protection}
The application may collect or handle user-submitted data, such as reports on sidewalk conditions or accessibility concerns. To ensure legal compliance and maintain public trust, the following principles must be upheld:

\begin{itemize}
    \item \textbf{Data Minimization}: Collect only the data strictly necessary for the functionality of the platform.
    \item \textbf{Anonymity and Pseudonymity}: Avoid collecting identifiable information unless required. If user accounts are needed, anonymize personal details where feasible.
    \item \textbf{Compliance with Regulations}: All data handling must comply with relevant laws, including the \textit{California Consumer Privacy Act (CCPA)} and, where applicable, the \textit{General Data Protection Regulation (GDPR)}.
    \item \textbf{Consent and Transparency}: Users must be informed of what data is being collected, how it will be used, and consent must be obtained where required.
    \item \textbf{Right to Access and Delete}: Users must be able to view, edit, and delete their data upon request.
\end{itemize}

\subsection{Data Storage and Security}
\begin{itemize}
    \item \textbf{Secure Storage}: All user data must be stored securely using encryption standards (e.g., AES-256 for data at rest, HTTPS/TLS for data in transit).
    \item \textbf{Access Control}: Only authorized personnel or services should have access to user data. Role-based access must be implemented.
    \item \textbf{Data Retention Policy}: Establish clear rules about how long data is retained and when it is purged to avoid unnecessary data accumulation.
\end{itemize}

\subsection{Ethical Considerations and Moral Dilemmas}
\begin{itemize}
    \item \textbf{Bias and Fairness}: Ensure that any decision-support features (e.g., prioritizing sidewalk repairs) do not unfairly disadvantage communities based on location, demographics, or socio-economic factors.
    \item \textbf{Accessibility}: The platform must serve all users equally, particularly people with disabilities, in accordance with the \textit{Americans with Disabilities Act (ADA)}.
    \item \textbf{Transparency and Accountability}: Any analysis, ranking, or scoring of sidewalk conditions must be explainable and documented to avoid opaque or biased decision-making.
    \item \textbf{Public Trust}: Misrepresentation of sidewalk data or failure to act on valid community reports can erode trust. Ethical handling of submitted reports and respectful communication with users are critical.
\end{itemize}

\newpage
\section{Glossary}

\begin{itemize}
    \item \textbf{SRS}: Software Requirements Specification. A document that describes the functionality and constraints of a system.
    \item \textbf{API}: Application Programming Interface. A set of routines, protocols, and tools for building software and applications.
    \item \textbf{GIS}: Geographic Information System. A framework for gathering, managing, and analyzing spatial and geographic data.
    \item \textbf{UI}: User Interface. The space where interactions between humans and machines occur.
    \item \textbf{UX}: User Experience. The overall experience of a user when interacting with a system or product.
\end{itemize}

\end{document}
